\documentclass[10pt,fleqn]{scrartcl}
%\documentclass[10pt]{article}
\usepackage[fleqn]{amsmath}
%\usepackage[latin1]{inputenc}
\usepackage[utf8]{inputenc}
\usepackage{amsmath}
\usepackage{amsfonts}
\usepackage{amssymb}
\usepackage[left=0.5in, right=0.5in, top=0.5in, bottom=0.75in]{geometry}
\usepackage{hyperref}
\usepackage{graphicx}
\usepackage{gensymb}

%\usepackage[compact]{titlesec}

\usepackage{mathpazo}

\usepackage{enumitem}
\usepackage{textcomp}
\usepackage{xcolor}

\urlstyle{same}
\hypersetup{
     urlcolor=blue,}

\setcounter{tocdepth}{4}

\newcommand{\school}[3]{
   \item \textsc{#1} \hfill {#2}\\
   #3\\
}

\begin{document}

\noindent
\textcolor{blue}{\LARGE{\textbf{Khalid Hossain}}} \\
\textcolor{blue}{\large{Leadership Computing Facility, Argonne National 
Laboratory}}
%\begin{flushright}
%  1650 NE Valley Road, Apt\# J2,\\
%  Pullman, Washington, 99163 \\
%  \textbf{\href{mailto:khossain4337@gmail.com}{khossain4337@gmail.com}} \\
%  \textbf{\href{mailto:mdkhalid.hossain@wsu.edu}{mdkhalid.hossain@wsu.edu}} \\
%  (509) 339-5978
%\end{flushright}
\noindent
\section*{\textcolor{blue}{Skills}}
\begin{itemize}
     \item \textbf{5+} years of experience of developing 
         \textbf{quantum simulation software} 
        using denisty functional theory (\textbf{DFT}) with 
        \textbf{python}.
    \item High performance computing \textbf{(HPC)} techniques (user level) to 
        perform one of the largest simulations of the dynamics in fermionic 
        systems using 
        \textcolor{blue}{\textbf{\href{https://gitlab.fizyka.pw.edu.pl/wtools/wslda}{wslda-toolkit}}}.
	\item Understanding of \textbf{superfluidity} (advanced) and 
        \textbf{superconductivity} (intermediate) at both 
        \textbf{phenomenological} and \textbf{microscopic} theory level.
    \item \textbf{Machine learning (ML)} techniques in calculating fission 
        properties for nuclei using \textbf{PyTorch}.
    \item Comfortable using \textbf{energy optimization} techniques to 
        determine initial states in superfluid systems. 
\end{itemize}

\noindent
\section*{\textcolor{blue}{Research Experiences}}
\noindent
\textbf{Postdoctoral Appointee} \hfill{January 2023 - Current}\\
Supervisor: Corey Adams \\
Mentor: Kyle Felker \\
Leadership Computing Facility, Argonne National Laboratory
\begin{itemize}
\item \textit{Deep Learning in Many-body Physics}\\
    We are working on calculating different ground state properties like 
        density of nucleons in small to medium sized nuclei employing 
        artificial neural network (ANN) based representation of the many-body
        wavefunction, and variational Monte-Carlo technique. The state-of-the-
        art calculations compute properties of nuclei up to total $6$ nucleons.
        Our goal is to extend these simulations using distributed computing and
        leadership class facilities for larger systems (up to $20$), providing
        us with insights into the fundamental nuclear interactions. 

\item \textit{Performance Analysis of Supercomputing Applications} \\
    Our aim is to develop performance metrics and analyze them for scientific
        applications at different levels of distributed computing (single vs. 
        multiple GPUs, nodes; medium to large scale distribution etc.) 
        across many available hardware platforms like Intel, Nvidia, AMD etc. 
        This will allow us to develop better programming strategies to operate
        leadership class supercomputers near the peak performance level.
\end{itemize}
\noindent
\textbf{Summer Intern} \hfill June 2021 - August 2021\\
Supervisors: Marc Verriere, Irene Kim, and Nicolas Schunck\\
Nuclear Data and Theory Group, Lawrence-Livermore National Laboratory
\begin{itemize}
\item \textit{Machine learning in nuclear theory}\\
    We are using machine learning techniques to determine nuclei's ground state 
    and fission properties across the nuclear chart. The project involves 
    training a deep neural network (more precisely, a convolutional denoising 
    autoencoder) to determine the Hartree-Fock-Bogoliubov solution from the 
    first iterations of HFBTHO, a FORTRAN nuclear Density Functional Theory 
    code actively developed in the Nuclear Data Group of LLNL. A properly 
    trained network could drastically reduce the number of iterations to obtain 
    self-consistent solutions of the HFB equations. The neural network is 
    conceived and trained using the python library PyTorch.    
\end{itemize}
\noindent
\textbf{Research Assistant}   \hfill May 2016 - May 2022 \\ 
Supervisor: Michael McNeil Forbes, Ph.D. \\
Department of Physics and Astronomy, WSU, Pullman, WA
\begin{itemize}
  \item \textit{'Negative-mass' hydrodynamics}\\
  		DFT implementation to simulate the dynamics of trapped $^{87}$Rb 
        Bose-Einstein condensate. This research led to the identification of 
        the mechanism behind the origin of effective mass and explained 
        observed phenomena in this bosonic superfluid system. In collaboration 
        with the 'Fundamental Quantum Physics' lab of Professor Peter Engels 
        at Washington State University. 
  \item \textit{Andreev-Bashkin effect (entrainment, dissipationless 
    superfluid drag)}\\
  		Developed an experimental protocol to directly detect bulk 3D 
        entrainment in the superfluid mixture of $^{174}$Yb (boson) and 
        $^{6}$Li (fermion). Entrainment has been predicted by A. F. Andreev 
        and E. P. Bashkin in 1975 and has not been experimentally observed ever 
        since. The experimental detection is expected to shed light on the 
        long-standing astrophysical mystery of neutron star 'glitch' -- 
        a sudden increase in the rotation rate of the star. Entrainment is 
        believed to play a crucial in the development of this effect. 
        This research is done in close collaboration with Professor Subhadeep 
        Gupta's lab at University of Washington.
  \item \textit{Rotating Quantum Turbulence}\\
  		Quantum turbulence is characterized by the dynamic interactions between 
        quantized vortices in superfluids. We are developing orbital-free DFT 
        models to study these interactions in a rotating system and validate 
        the results against more accurate TDDFT models like superfluid local 
        density approximation (SLDA). This validation is crucial to build 
        orbital-free density functionals to simulate the dynamics in 
        macroscopically large samples of fermionic superfluids as time 
        evolution of many orbitals for long times can be quite expensive 
        numerically. This research is done in collaboration with Professor 
        Gabriel Wlazłowski's group at Warsaw University of Technology, Poland.
  	\item \textit{Isotropic Quantum Turbulence}\\
  		In similar spirit to the rotating turbulence project, but performed for 
        isotropic  superfluid in a 3D box. Involves handling of much larger 
        data sets and HPC. Computing assignments and analysis will be performed 
        in SUMMIT and RHEA -- flagship supercomputing facilities maintained by 
        Oak Ridge National Lab. 
\end{itemize}

\noindent
\section*{\textcolor{blue}{Education}}
\begin{itemize}
\school{Washington State University}{Aug 2014 - May 2022}
            {Department of Physics and Astronomy, Pullman, WA}
 		\textbf{Ph.D. in Atomic Physics}  \\
		\textbf{Advisor: Michael McNeil Forbes, Ph.D.}
\school{University of Dhaka}{Jan 2012 -  July 2013}
            {Dhaka, Bangladesh}
 		\textbf{Master of Science in Theoretical Physics}  
\end{itemize}

%\newpage

\noindent

\section*{\textcolor{blue}{Publications}}
\begin{itemize}
\item ``Negative-Mass Hydrodynamics in a Spin-Orbit-Coupled Bose-Einstein 
        Condensate''\\
	    M. A. Khamehchi, \textbf{K.H.}, M. E. Mossman, Y. Zhang, Th. Busch, 
        M. M. Forbes, and P. Engels.\\
	   \href{https://journals.aps.org/prl/abstract/10.1103
                /PhysRevLett.118.155301}
	   {\textcolor{blue}{Phys. Rev. Lett. \textbf{118}, 155301 (2017)}}
\item ``Rotating Quantum Turbulence in the Unitary Fermi Gas''\\
        \textbf{K.H.}, K. Kobuszewski,  M. M. Forbes, P. Magierski, K. Sekizawa,
		and G. Wlazłowski.\\
      \href{https://journals.aps.org/pra/abstract/10.1103/PhysRevA.105.013304}
        {\textcolor{blue}{Phys. Rev. A \textbf{105}, 013304 (2021)}}
\item ``Detecting Entrainment in Fermi-Bose Mixture''\\
	  \textbf{K.H.}, S. Gupta, and M. M. Forbes.\\
        \href{https://journals.aps.org/pra/abstract/10.1103/
        PhysRevA.105.063315}{\textcolor{blue}
        {Phys. Rev. A \textbf{105}, 063315 (2022)}}
	  
	  
\end{itemize}

\noindent
\section*{\textcolor{blue}{Talks}}
\begin{itemize}
\item ``Homogeneous and Isotropic Turbulence across the BEC-BCS Crossover " \\
			APS DAMOP Virtual Meeting \hfill June, 2021
\item ``Quantum Turbulence: Generation and Decay in Bosonic and Fermionic Superfluids" \\
			APS DAMOP Virtual Meeting \hfill June, 2020
\item ``Energy Dissipation and Vortex Recombination in the Unitary Fermi Gas''\\
             APS DAMOP meeting   \hfill May, 2019
\item  ``Detecting Entrainment in Fermi-Bose Mixtures''\\
		APS DAMOP meeting        \hfill May, 2018
\item ``Negative-mass Hydrodynamics in a spin-orbit coupled Bose-Einstein 
	Condensate'' \\
      APS DAMOP meeting          \hfill June 2017
\end{itemize}

\noindent
\section*{\textcolor{blue}{Awards and Honors}}
\begin{itemize}
  \item Radziemski Fellowship   \hfill June 2018 \\
  		College of Arts and Sciences, Washington State University
%  \item Travel Grant, American Physical Society (APS) \hfill June 2017\\
% 		48$^{th}$Annual conference \\
%        Division of Atomic, Molecular and Optical Physics (DAMOP).  		
\end{itemize} 

\noindent
\section*{\textcolor{blue}{Teaching and Mentoring Experience}}
\textbf{Lecturer} \\ 
\textit{ Washington State University}                               \hfill{Fall 2022}
\begin{itemize}
    \item Taught ``Physics 150: Physics and Your World".
\end{itemize}
\textbf{Graduate Student Instructor}            \hfill{Spring 2018, 2022}
\begin{itemize}
\item Instructor for ``Physics 150".
\end{itemize}                                                

\textbf{Physics Teaching Assistant (TA)}                                            \hfill Aug 2014 - Dec 2017
\begin{itemize} 
\item Lab instructor for introduction to physics (Physics 102, 201, 202).
\end{itemize}
\noindent

\section*{\textcolor{blue}{Training and Development}}
\begin{itemize}
\item Attended workshop on ``\href{http://www.int.washington.edu/PROGRAMS/19-1a/}
{\textcolor{blue}{Quantum Turbulence: Cold Atoms, Heavy Ions, and Neutron Stars}}'' \hfill March 2019\\
Institute of Nuclear Theory \\ 
University of Washington, Seattle
\end{itemize}
%\noindent
%\section*{\textcolor{blue}{Media Coverage}}
%\begin{itemize}
%    \item \textbf{BBC} \quad 
%		\href{https://www.bbc.com/news/science-environment-39642992}
%                      {\textcolor{blue}{Physicists observe 'negative mass'}}
%	\item \textbf{The Guardian} \quad 
%		\href{https://www.theguardian.com/science/2017/apr/19/scientists-have-
%		created-a-fluid-with-negative-mass-but-what-does-it-tell-us}
%		{\textcolor{blue}{Scientists have created a fluid with negative mass -- 
%		but what does it tell us}}
%	\item \textbf{APS News}\quad 
%		\href{https://www.aps.org/publications/apsnews/201707/disperse.cfm}
%		{\textcolor{blue}{To Disperse, or To Not Disperse: Debating 
%		``Negative Mass''}}
%\end{itemize}

%\noindent
%\section*{\textcolor{blue}{References}}
%	\begin{tabular}{lll}
%		\hspace{5pt} Professor Michael Forbes, Ph.D.  & 
%		\hspace{5pt} Professor Gabriel Wlaz\l{}owski, Ph.D. & 
%		\hspace{5pt} Professor Peter Engels, Ph.D. \\
%		\hspace{5pt} Dept. of Physics and Astronomy & 
%		\hspace{5pt} Faculty of Physics  & 
%		\hspace{5pt}  Dept. of Physics and Astronomy \\
%		\hspace{5pt} Washington State University   & 
%		\hspace{5pt} Warsaw University Technology & 
%		\hspace{5pt} Washington State University \\
%		\hspace{5pt} \small{\href{mailto:m.forbes@wsu.edu}
%		{\textcolor{blue}{m.forbes@wsu.edu}}} & 
%		\hspace{5pt}\small{\href{mailto:gabriel.wlazlowski@pw.edu.pl}
%		{\textcolor{blue}{gabriel.wlazlowski@pw.edu.pl}}} & 
%		\hspace{5pt}\small{\href{mailto:engels@wsu.edu}
%		{\textcolor{blue}{engels@wsu.edu}}} \\
%	\end{tabular}
 


\end{document}
