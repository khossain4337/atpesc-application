\documentclass{article}

\usepackage{geometry}
\usepackage{amsmath, amssymb}

\title{Blurb for the letter}

\begin{document}

\maketitle

Khalid has started as a postdoc at ALCF on January, 2023. He got his PhD in
physics from Washington State University. In his graduate school years he 
worked on developing scientific software to simulate quantum dynamics of 
superfluid systems. These were density functional theory (DFT) implementations
using Python. A significant part of his dissertation work was dedicated to
reliably generate and study turbulent states in fermionic superfluid systems. 
Apart from being interesting in their own right as quantum simulators, 
superfluid dynamics have applications in nuclear astrophysics, specially in 
discussing pulsar glitches which are not fully understood yet.

Simulating fermionic quantum systems in experimentally relevant volumes is 
computationally challenging because of the size of the associated states. 
Therefore these become interesting test cases for leadership class facilities
like ALCF, OLCF etc. Subsequently Khalid worked as a trainee graduate student
in an ALCC project where he has performed large scale simulations in Summit at
Oak Ridge.

Briefly, the computational task was to diagonalize a Hamiltonian matrix using
self-consistent iterative methods to get the minimum energy state of the 
superfluid system within desired accuracy. Once the initial state has been 
acquired, the time evolution is performed to produce a turbulent state and 
study it's decay and other dynamic behaviors. It is imperative to have a good
seed state for iterative solver to produce a promising initial state for the 
dynamics. Khalid was responsible for making these seed states, writing the job
scripts and ensuring the successful completion of a simulation and optimal 
usage of their allocation time to make a campaign worth significant scientific 
value. He has actively contributed in developing effective models for these
system to analyze the data from direct numerical solutions of the partial 
differential equations that describe the dynamics.

During the course of the ALCC project, Khalid was exposed to performance 
analysis and scaling behavior of a scientific application. Although he has not
dived deep, this is where he wants to start his future endeavors of 
becoming a computational scientist. Various training tracks at ATPESC, ranging 
from MPI and OpenMP
based distributed computing, advanced debugging and benchmarking to scientific 
software designing will give him a boost forward.

He has also worked as a summer intern at Lawrence-Livermore where he
has worked with deep learning (DL) methods applied to nuclear fission problems. 
Through this work he has developed interest in DL methods, and he is currently
working on a project to calculate different static properties of a nucleus 
using such methods. Attending sessions at ATPESC on DL methods and training 
large models in a distributed computing environment will significantly increase 
his chance of success.

Khalid wants to pursue a career as a computational scientist, where he would
work with domain scientists to enable them in leveraging the resources 
available at leadership class facilities to tackle important problems. He has 
shown promise and it would be a great opportunity for him at the early stage
of his career to have exposure to new ideas and trends discussed at ATPESC 
while establishing connections with leading experts in the field.



\end{document}
